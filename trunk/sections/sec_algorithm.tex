\section{Overview of the CSC Local Trigger Algorithms}
\label{sec:csc_algo}

An improved ME1/1 local trigger algorithm to perform sophisticated background rejection is developed for operation under high pile-up running conditions.

\subsection{Results of Improvements in the ME1/1 Local Trigger Algorithm}
\label{sec:SLHC_algo_results}

This chapter presents results of the study of effects of individual improvements described in Sec.~\ref{sec:SLHC_algo} on the ALCT, CLCT, and LCT reconstruction efficiencies.

The study is performed with Monte Carlo simulation of double muon events mixed with PU400 events, where the simulation includes GEN, SIM, DIGI, L1 steps.

Three are three baseline configurations of L1 step used in this study:
\begin{itemize}
	\item Baseline 1: SLHC configuration, where the maximum set of improvements is turned off bringing it to 2007 configuration as close as possible. 
	There are only two differences between Baseline 1 and 2007 configurations: separate treatments of ME1/1a and ME1/1b, and unganging cathode strips in ME1/1a;
	\item Baseline 2: Baseline 1 configuration with all improvements on the ALCT and CLCT processors level turned on;
	\item Baseline SLHC: best CSC4 configuration itself.
\end{itemize}

All algorithm improvements divided into three groups and studied with improvements in the given group turned on one by one on top of each other
\begin{itemize}
	\item ALCT processor level
	\item CLCT processor level
	\item TMB level
\end{itemize}

In the first two groups the L1 step configuration gradually changes from Baseline 1 to Baseline 2 configuration, in the last one --- from Baseline 2 to SLHC configuration.